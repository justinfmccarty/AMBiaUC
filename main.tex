\documentclass{article}
\usepackage[utf8]{inputenc}

\title{Adaptation and Mitigation for Buildings in an Urban Context}
\author{Justin McCarty}
\author{Adam Rysanek}
\date{April 2021}

\begin{document}

\maketitle

\section{Abstract}
British Columbia is a rapidly urbanizing province in Southwestern Canada. As well the built environment is on the cusp of needing space cooling solutions to keep occupants comfortable as well as confront high heat events and contend with increasingly common periods of urban wildfire smoke. Simultaneously the province has set out targets for decarbonizing the building stock by requiring all new buildings to be net-zero energy ready starting in 2032 and setting out incentive programs for retrofitting the existing carbon stock, primarily through fuel switching initiatives. BC and its primary urban regions have been the target of many studies at the scale of the building focused on various adaptation and mitigation measures. However, solutions for entire urban contexts are rarely evaluated from a bottom up perspective. In this study we look over the results of a comprehensive urban scale energy simulation for a representative mid-density development in southwestern BC. 

\section{Introduction}
Atmospheric concentrations of carbon dioxide are just over 418 parts per million, while recent estimates from the \citet{noauthor_state_2020} situate the Northern Hemisphere’s land and ocean surface temperature anomaly at 1.29°C from baseline (1961-1990). While this warming level is not considered the consistent level of \textit{global warming} at 1.5°C, it does indicate a dangerous warming trend facing human and non-human systems. 

Rising temperature are likely to have cuased an increasse in heat-realted death

Climate change impacts are now being recognized as burdens on economic and social systems \citep{frame_climate_2020}. Implementing adaptation strategies within the built environment is necessary to increase social resilience, protect safety and life, and prevent economic damage with the potential to bolster all of the above \citep{schunemann_mitigation_2020}. Simultaneously, the mitigation of present emissions, the abatement of possible future emissions, and the sequestration of atmospheric carbon is necessary to curtail increasingly dangerous and catastrophic levels of warming. This process of carbon drawdown is very familiar within studies of the built environment with strategies posed for transportation, buildings, agriculture, forestry, and other land-uses. Mitigation and adaptation strategies though are rarely evaluated simultaneously in the context of urban development.


\end{document}
